\section{Related Work}

\subsection{Voice Input}
Voice input is a popular input method available on most mobile devices. Currently, smartwatches also use voice input. Smartwatches on the market include the Apple Watch, Moto G, Sony SmartWatch 3, Pebble, Samsung Gear S, and Huawei Watch \cite{huawei-watch}. All these smartwatches support voice input for performing a variety of actions. Voice input provides a hand-free and high-speed text entry method to mobile device users \cite{voice-input-mobile,android-speech-recognition}. However, voice input may not be socially acceptable in certain situations. For example, in public spaces, it is not common to see people using their device with voice control. Furthermore, users may find voice input unreliable in regards to speech recognition accuracy. Voice recognition is not perfect, especially when the quality of the users’ voice or accent and ambient sounds in the environment negatively affect the output. Therefore, to make text entry reliable across different scenarios, an alternative text input method is necessary for smartwatches.

\subsection{Software Keyboard}
Since smartphone users are accustomed to using touchscreen keyboards, software keyboard designers naturally want to find ways of fitting keyboards onto smartwatches.

\subsubsection{ZoomBoard}
ZoomBoard \cite{zoomboard}, designed by Carnegie Mellon University \& Friendship House, is designed for small mobile devices. ZoomBoard maps QWERTY soft-keyboard onto the display. ZoomBoard zooms in on the keyboard area that the user tapped on. On this now enlarged section of the keyboard, the user can tap on a specific button to type the selected character. The keyboard will then zoom out to its original size. Since zoom navigation is common on smartphones, users will find ZoomBoard easy to learn as it does not require users to memorize any kind of special input techniques. This text entry method is designed for small screen devices and it is shown to be more suitable for small screens instead of medium or large screen sizes.\cite{text-entry-on-small-qwerty}

\subsubsection{Swipeboard}
Swipeboard \cite{swipeboard}, by Autodesk Research, is another software keyboard made for small devices. The 2 swipe or tap text input method is shown to outperform ZoomBoard after less than a 2 hour training, and this approach allows for fast eyes-free text entry. The Swipeboard input method consists of two parts: the first swipe or tap is to decide which section of the keyboard to enlarge, and the second swipe or tap is to select the target character within that region. Because this kind of input method relies on swipe motions more than tap motions, users may have to spend some time learning how to use this new input technique. Hence, the initial speed of text entry on Swipeboard is slightly slower than that of Zoomboard. However, Autodesk Research claims that users can practice in order to achieve higher typing speeds.

\subsubsection{FlickKey}
FlickKey \cite{flickkey}, a keyboard originally designed for smartphones, consists of a keyboard divided into six sections each containing nine characters. Users can swipe on the button towards a character to type out one of the characters located on the edges of the button, or tap on the button to select the character in the middle of the button. The idea of FlickKey is analogous to that of \papertitle, but since it was originally designed for smartphones, it contains 9 x 6 = 54 keys on the screen. This number of keys is far more than necessary for simple 26 character English text entry. Thus the effective size of each button is smaller than expected for an optimized keyboard. In addition, our user study shows that a button designed for 8 different directional swipe and 1 tap gestures would cause problems with usability.

\subsubsection{MessagEase}
MessagEase keyboard \cite{messageease}, by Exideas, has a layout originally designed for smartphones. It uses swipe or tap gestures to increase the effective size of buttons. It also takes Fitts' law into consideration. The most commonly used characters A, N, I, H, O, R, T, E, and S are placed in the center of the nine buttons. Other less commonly used characters are placed around the main characters. Hence, users can tap on one of the nine buttons to type out the most commonly used characters, or swipe on the buttons to select the surrounding, less commonly used characters. The MessagEase also has problems similar to FlickKey: the square layout of MessagEase would waste space on the sides if we were to put MessagEase on a smartwatch screen with a square shaped watch face.

\subsubsection{Japanese Smartphone Keyboard}
The most common Japanese smartphone soft-keyboard \cite{japanese-smartphone-keyboard}, found natively in iOS and Android, uses a layout where buttons are organized by Hiragana (Japanese alphabet) groups. Characters starting with the same vowel are grouped together in one button. Users can tap or swipe on one of these buttons to select the character for the desired syllable. Native Japanese speakers may find this keyboard easy to use since they naturally know the phonetic system underlying Hiragana characters.

% \subsection{Reduced keyboard Methods with Autocorrection and Autocomplete}
% There are many work try to improve soft keyboards on smartphones by including two functionalities: autocorrection and autocomplete.\cite{} Autocorrection helps users to automatically correct their spelling mistakes as they type. Autocomplete, on the other hand, displays three or four suggested words before the users finish typing a word. The users can tap on one of these suggested words to automatically type it out for them. General mobile users are accustomed to using both autocorrection and autocomplete features on their software keyboards to assist them in typing on smartphones. Autocorrection and autocomplete features can be implemented with any soft keyboards in order to the amount of time to type out phrases. 

% It is possible integrating autocorrection and autocomplete features in \papertitle\ and further help the users in text entry. In this paper, we only focus on \papertitle\ itself. Our scope is not including the integration of autocorrection an autocomplete.