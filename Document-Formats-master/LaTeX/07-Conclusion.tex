\section{Limitation and Conclusion}
In this work, we have focused on creating a one swipe or one tap keyboard for smartwatches. We have first discussed the 3 qualities of optimized keyboards. These requirements outline the design for \papertitle, and allowed us to conduct a series of user studies to narrow down the numbers of possible design options. We have implemented \papertitle\ 4 and \papertitle\ 5, both optimized for WPM, error rate, rating of difficulty and user preference. The result of \papertitle\ shows a 55\% improvement in WPM and a 44\% decrease in error rate compared to Swipeboard. We have discovered that the QWERTY keyboard layout does not provide significant advantages to small swipe-based keyboards according our experiment. This knowledge could be helpful for future keyboard designs on small devices since keyboard designers are not forced to distort their keyboard designs to accommodate the QWERTY layout.

There is an important design principle that emerges from our user study. A consistent design is crucial for intuitive, easy to use, and low error rate keyboard layouts.

We have focused our work on keyboard design for smartwatches. As suggested in the work of Luis etals. \cite{text-entry-on-small-qwerty}, designing a suitable keyboard depends on the screen size. For a smaller keyboard size, we may find that an even higher swipe direction number \papertitle\ such as \papertitle\ 6 or even \papertitle\ 9 may outperform \papertitle\ 4 or 5. This is because as the keyboard space becomes smaller, the button size will shrink accordingly, and eventually the size of a button would become smaller than the acceptable minimum size, leading to a drastic increase in input errors. Note that the error rate increases more slowly when increasing the number of swipe directions compared to shrinking the button size below a certain size limit.

Modern soft keyboards use word prediction and auto-correction to improve performance. Although we did not take this effect into consideration in this paper, it is possible to further improve \papertitle\ by integrating these features in future works.

We believe that our work provides a well-designed keyboard for smartwatches that enables intuitive, fast, and low error text entry. Smartwatch manufacturers can implement \papertitle\ on smartwatches, along with other input methods, such as voice input, to provide more choices for the users. Furthermore, this work provides a new understanding of swipe resolution and character arrangement on small devices. This information could be helpful for future researchers.
